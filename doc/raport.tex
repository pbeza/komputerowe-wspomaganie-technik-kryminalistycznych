\documentclass[]{article}
\usepackage{polski}
\usepackage[utf8]{inputenc}
\usepackage{hyperref}
\usepackage{marginnote}
\usepackage{enumerate}
\usepackage{graphicx}
\graphicspath{{img/}}

\DeclareGraphicsExtensions{.pdf,.png,.jpg}

%opening
\title{KWTK - EigenFaces}
\author{Raport z testów}

\begin{document}
\maketitle


\subsection*{Zespół projektowy:}
	inż. Patryk Bęza \newline
	inż. Krzysztof Małaśnicki \newline
	inż. Marek Kozak
	
\newpage


\section[Testy]{Propozycje testów do przeprowadzenia}
\label{sec:test}

\subsection{Testy logowania}
Seria testów ma przebadać działanie połączenia aplikacji z bazą.
\subsubsection{Poprawne dane}
Test polega na typowym logowaniu za pomocą danych root:toor, przy poprawnym działaniu bazy PostgreSQL.
\subsubsection{Niepoprawne dane}
Test polega na próbie logowania do aplikacji korzystając z niepoprawnych danych, przy poprawnym działaniu bazy PostgreSQL.
\subsubsection{Brak serwera logowania}
Test polega na próbie logowania podczas nieaktywności bazy danych.

\subsection{Testy algorytmu}
Celem serii testów jest analiza celności stosowanego algorytmu oraz jego słabych stron.
\subsubsection{Dane z bazy}
Po załadowaniu bazy rekordów wybierane jest jedno ze zdjęć zawartych w bazie celem poszukiwania.
\subsubsection{Dane lekko zmodyfikowane}
Seria testów polegających na wyszukiwaniu graficznie zmodyfikowanych poprzez dorysowanie elementów operacje macierzowe lub filtry graficzne.
\subsubsection{Dane ekstremalnie zmodyfikowane}
Seria testów polegających na wyszukiwaniu wzorców samej sylwetki, jej negatywu oraz twarzy wklejonej w sylwetkę innego profilu.
\subsubsection{Dane w niewłaściwym formacie}
Test polega na zapytaniu o profil o rozmiarze niezgodnym z bazą lub niewłaściwym typie pliku.
\subsubsection{Walidacja krzyżowa}
Podstawowy test walidacji krzyżowej, polegającym na wyrzuceniu zdjęcia z fazy nauki, a potem próbie jego wyszukiwania.

\section[Wyniki]{Wyniki z przeprowadzonych testów}
\subsection{Testy logowania}
\subsubsection{Poprawne dane}
Pomyślne zalogowanie prowadzi do kolejnego okna aplikacji.
\subsubsection{Niepoprawne dane}
Dla niepoprawnych danych logowania (user: abc, pass: edf) program zwraca komunikat o błędzie logowania i kończy działanie aplikacji.
\subsubsection{Brak serwera logowania}
Wyłączono serwisy postgresql, próbowano uruchomić aplikację. Dostajemy komunikat "Java exception has occured" i następuje zakończenie aplikacji.

\subsection{Testy algorytmu}

\subsubsection{Dane z bazy}
Dla obrazów z bazy danych aplikacja zwraca dokładnie zdjęcie które było już w bazie danych z distance = 0.0. Kolejne wyniki wskazują na tą samą osobę w odległości ~11k
\subsubsection{Dane lekko zmodyfikowane}
\begin{itemize}
	\item subject03.glasses.1, znaleziony poprawny wynik, dystans=1252, kolejne zdjęcia dystans ~12k
	\item subject07.glasses.2 znaleziony poprawny wynik, dystans~10k, kolejne zdjęcia dystans ~12k
	\item subject11.rightlight.2 wynik najlepszy niepoprawny, dystans~15k, dystans od pierwszego zgodnego ~15k
	\item subject13.normal.2 w wynikach brak poszukiwanej osoby, dystans od najbliższego zdjęcia ~18.5k
	\item subject15.wink.1 wynik najlepszy niepoprawny, dystans~18.9k, dystans od pierwszego zgodnego ~19.3k
\end{itemize}
\subsubsection{Dane ekstremalnie zmodyfikowane}
\begin{itemize}
	\item białe tło, kontur osoby wypełniony kolorem czarnym, dystans do najbliższego wyniku ~17k 
	\item czarne tło, kontur osoby wypełniony kolorem białym, dystans do najbliższego wyniku ~46k
	\item podmiana części twarzy subject01.glasses (na generała Franco), pierwotne zdjęcie znalezione poprawnie, dystans ~6.8k 
\end{itemize}
\subsubsection{Dane w niewłaściwym formacie}
\begin{itemize}
	\item Niewłaściwy rozmiar pliku, błąd wyłapany na etapie poszukiwana w bazie, wyświetlony komunikat o błędzie biblioteki
	\item Niewłaściwy typ pliku: aplikacja ignoruje wybrany plik.
	\item Poprawny typ, złe rozszerzenie: typ MIME jest jako zgodny z rozszerzeniem
\end{itemize}
\subsubsection{Walidacja krzyżowa}
Walidacja krzyżowa przebiega poprawnie, znajduje poprawne osoby zwracając w odległości poniżej 10k.

\section[Wnioski]{Dodatkowe uwagi dotyczące aplikacji}
\label{sec:more}
Klika dodatkowych uwag dot. funkcjonalności oraz działania aplikacji:
\begin{itemize}
\item Aplikacja pozwala na eksport modelu w formie xml, co jednak zdaje się być nadmiarowe, z uwagi na fakt, że nie pozwala na import modelu.
\item W oknie wyników nie można oglądać równocześnie podglądu wyszukiwanej twarzy oraz wyników wyszukiwania. Ponadto po przejściu do podglądu wyniku zdaje się być niemożliwy powrót do poszukiwanego profilu.
\item W oknie podglądu po zakończeniu wyszukiwania zwracana jest wartość \textbf{Total images} na 0.
\item Aplikacja nie zapamiętuje folderu szukanych zdjęć, co bywa uciążliwe.
\end{itemize}


\newpage
\tableofcontents
\end{document}

